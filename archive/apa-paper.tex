\documentclass[man]{apa6}

\usepackage{amssymb,amsmath}
\usepackage{ifxetex,ifluatex}
\usepackage{fixltx2e} % provides \textsubscript
\ifnum 0\ifxetex 1\fi\ifluatex 1\fi=0 % if pdftex
  \usepackage[T1]{fontenc}
  \usepackage[utf8]{inputenc}
\else % if luatex or xelatex
  \ifxetex
    \usepackage{mathspec}
    \usepackage{xltxtra,xunicode}
  \else
    \usepackage{fontspec}
  \fi
  \defaultfontfeatures{Mapping=tex-text,Scale=MatchLowercase}
  \newcommand{\euro}{€}
\fi
% use upquote if available, for straight quotes in verbatim environments
\IfFileExists{upquote.sty}{\usepackage{upquote}}{}
% use microtype if available
\IfFileExists{microtype.sty}{\usepackage{microtype}}{}

% Table formatting
\usepackage{longtable, booktabs}
\usepackage{lscape}
% \usepackage[counterclockwise]{rotating}   % Landscape page setup for large tables
\usepackage{multirow}		% Table styling
\usepackage{tabularx}		% Control Column width
\usepackage[flushleft]{threeparttable}	% Allows for three part tables with a specified notes section
\usepackage{threeparttablex}            % Lets threeparttable work with longtable

% Create new environments so endfloat can handle them
% \newenvironment{ltable}
%   {\begin{landscape}\begin{center}\begin{threeparttable}}
%   {\end{threeparttable}\end{center}\end{landscape}}

\newenvironment{lltable}
  {\begin{landscape}\begin{center}\begin{ThreePartTable}}
  {\end{ThreePartTable}\end{center}\end{landscape}}

  \usepackage{ifthen} % Only add declarations when endfloat package is loaded
  \ifthenelse{\equal{\string man}{\string man}}{%
   \DeclareDelayedFloatFlavor{ThreePartTable}{table} % Make endfloat play with longtable
   % \DeclareDelayedFloatFlavor{ltable}{table} % Make endfloat play with lscape
   \DeclareDelayedFloatFlavor{lltable}{table} % Make endfloat play with lscape & longtable
  }{}%



% The following enables adjusting longtable caption width to table width
% Solution found at http://golatex.de/longtable-mit-caption-so-breit-wie-die-tabelle-t15767.html
\makeatletter
\newcommand\LastLTentrywidth{1em}
\newlength\longtablewidth
\setlength{\longtablewidth}{1in}
\newcommand\getlongtablewidth{%
 \begingroup
  \ifcsname LT@\roman{LT@tables}\endcsname
  \global\longtablewidth=0pt
  \renewcommand\LT@entry[2]{\global\advance\longtablewidth by ##2\relax\gdef\LastLTentrywidth{##2}}%
  \@nameuse{LT@\roman{LT@tables}}%
  \fi
\endgroup}


  \usepackage{graphicx}
  \makeatletter
  \def\maxwidth{\ifdim\Gin@nat@width>\linewidth\linewidth\else\Gin@nat@width\fi}
  \def\maxheight{\ifdim\Gin@nat@height>\textheight\textheight\else\Gin@nat@height\fi}
  \makeatother
  % Scale images if necessary, so that they will not overflow the page
  % margins by default, and it is still possible to overwrite the defaults
  % using explicit options in \includegraphics[width, height, ...]{}
  \setkeys{Gin}{width=\maxwidth,height=\maxheight,keepaspectratio}
\ifxetex
  \usepackage[setpagesize=false, % page size defined by xetex
              unicode=false, % unicode breaks when used with xetex
              xetex]{hyperref}
\else
  \usepackage[unicode=true]{hyperref}
\fi
\hypersetup{breaklinks=true,
            pdfauthor={},
            pdftitle={How engagement during out-of-school time STEM programs promotes the development of youths' interest in STEM domains},
            colorlinks=true,
            citecolor=blue,
            urlcolor=blue,
            linkcolor=black,
            pdfborder={0 0 0}}
\urlstyle{same}  % don't use monospace font for urls

\setlength{\parindent}{0pt}
%\setlength{\parskip}{0pt plus 0pt minus 0pt}

\setlength{\emergencystretch}{3em}  % prevent overfull lines


% Manuscript styling
\captionsetup{font=singlespacing,justification=justified}
\usepackage{csquotes}
\usepackage{upgreek}



\usepackage{tikz} % Variable definition to generate author note

% fix for \tightlist problem in pandoc 1.14
\providecommand{\tightlist}{%
  \setlength{\itemsep}{0pt}\setlength{\parskip}{0pt}}

% Essential manuscript parts
  \title{How engagement during out-of-school time STEM programs promotes the
development of youths' interest in STEM domains}

  \shorttitle{Engagement in STEM}


  \author{Joshua Rosenberg\textsuperscript{1}, Patrick Beymer\textsuperscript{1}, \& Jennifer Schmidt\textsuperscript{1}}

  % \def\affdep{{"", "", ""}}%
  % \def\affcity{{"", "", ""}}%

  \affiliation{
    \vspace{0.5cm}
          \textsuperscript{1} Michigan State University  }

  \authornote{
    This paper is to be presented at the 2018 Annual Meeting of the American
    Educational Research Association, New York, NY. Correspondence regarding
    this manuscript may be directed to Joshua Rosenberg
    (\href{mailto:jrosen@msu.edu}{\nolinkurl{jrosen@msu.edu}}). This
    material is based upon work supported by the National Science Foundation
    under Grant No: DRL-1421198. Any opinions, findings, conclusions, or
    recommendations expressed in this material are those of the authors and
    do not reflect the views of the National Science Foundation.
    
    Correspondence concerning this article should be addressed to Joshua
    Rosenberg, Postal address. E-mail:
    \href{mailto:jrosen@msu.edu}{\nolinkurl{jrosen@msu.edu}}
  }


  \abstract{Enter abstract.}
  \keywords{Engagement \\

    \indent Word count: X
  }





\usepackage{amsthm}
\newtheorem{theorem}{Theorem}
\newtheorem{lemma}{Lemma}
\theoremstyle{definition}
\newtheorem{definition}{Definition}
\newtheorem{corollary}{Corollary}
\newtheorem{proposition}{Proposition}
\theoremstyle{definition}
\newtheorem{example}{Example}
\theoremstyle{definition}
\newtheorem{exercise}{Exercise}
\theoremstyle{remark}
\newtheorem*{remark}{Remark}
\newtheorem*{solution}{Solution}
\begin{document}

\maketitle

\setcounter{secnumdepth}{0}



Introduction

Recently, out-of-school-time programs focusing on science, technology,
engineering, and mathematics (STEM) have proliferated in order to combat
declines in STEM interest during adolescence (Brophy, 2008; National
Academy of Engineering and National Research Council, 2014; OECD, 2016)
and to meet the demands of a rapidly growing STEM workforce (Fayer et
al., 2017). Though many have argued that contexts for learning outside
of the school setting have an important role to play in youths'
development of interest (Bell, Lewenstein, Shouse \& Feder, 2009; Hidi
\& Renninger, 2006), relatively little is known about whether and how
youths' interest develops in such contexts. Contemporary motivational
theory suggests that interests emerge from the interactions of an
individual in a particular environment, rather than residing completely
within the individual or the environment (Hidi et al., 2004; Prenzel,
1992). Thus it is important to understand the ways that individuals
engage with STEM-focused environments in order to understand how
STEM-related interests may emerge. The purpose of this paper, then, is
to examine whether and how youths' sustained engagement in summer STEM
programming promotes the development of interest in STEM domains. To
explore this issue, we employ unique methods of data collection and
analysis that allow us to examine the effect of sustained
multidimensional engagement in STEM program on the development of
youth's interest over time.

\subsection{Out-of-school Time as a Context for Interest Development in
STEM}\label{out-of-school-time-as-a-context-for-interest-development-in-stem}

Out-of-school time is an overarching term used to refer to summer
programs as well as after-school programs where youth participate in
voluntary learning (Lauer et al., 2006). These programs have been touted
for their ability to provide youth with mentors as well as enable youth
to further develop their identity (Hirsch et al., 2010). Recently, there
has been an influx of out-of-school time programs focused on STEM
domains: Such programs are developed with the purpose of increasing
youth interest in STEM careers (Dabney et al., 2012; Elam et al., 2012).
An emerging body of research provides some evidence that such programs
can be effective in achieving this aim. One study showed that youth who
attend out-of-school time STEM programs have a higher likelihood of
choosing STEM career paths (Dabney et al., 2012). Some have suggested
that out of school time learning environments might be especially
effective at developing interest in STEM (or other) domains because
these programs are free from typical school constraints such as set
curricula, and therefore are able to focus their time on engaging
activities (Renninger, 2007). However, little is known about how one's
level of actual engagement in informal learning settings such as summer
STEM programs impacts the development of interest. It is critical to
understand whether and how interest can develop in these informal
programs and whether the degree of one's engagement in these programs
may contribute to this development.

\subsection{Interest Development}\label{interest-development}

Hidi and Renninger (2006) describe individual interest as \enquote{a
relatively enduring disposition to re-engage particular contents over
time} (p.~111). They, and other contemporary theorists frame interest as
the product of the interaction between a person and particular content,
and as such, interest is content specific (Hidi et al., 2004; Krapp,
2000; Prentzel, 1992; Renninger \& Wozniak, 1985). A person's enduring
interest in STEM content (denoted as individual interest by Hidi and
Renninger (2006)), for example, is the result of the way that person has
interacted with STEM ideas, practices and content over time. These
interactions will be shaped by features of the environments in which
such interactions take place, by other individuals who facilitate these
interactions, and by a person's own efforts, such as their self
regulation and the way they choose to engage with particular content
(Renininger, 2000; Renninger \& Hidi, 2002; Sansone \& Smith, 2000). A
more enduring individual interest can developed as a result of the
repeated experience of situational interest -- a more fleeting interest
in a specific activity or task that is typically characterized by
immediate positive affective reactions to the immediate activity at
hand, but may or may not be sustained over time.

Facilitating the development of youths' enduring individual interest in
STEM fields has been an explicit goal of many of the summer STEM
programs that have proliferated throughout the united states in recent
years, including the nine programs that are the focus of this study.
These programs are typically designed to support interest development by
presenting STEM content in novel ways (i.e., by appealing to situational
interest), by involving local experts in the field in programming, and
by allowing opportunities for deeper and more focused interaction with
STEM content without many of the constraints, distractions, and
pressures of traditional school environments. As such, summer STEM
programs often have many supports built in for interest development.
Youth will benefit differently from these supports though, depending on
how they choose to engage with the STEM content within the confines of
these programs. Youth who engage with STEM content in different ways may
differently support the development of their own interest in the
content. This study is aimed at exploring the degree to which youth can
support their own interest development through their engagement in
summer STEM programs.

Interest development is presumed to involve cognitive, affective, and
behavioral processes, such that a more enduring personal interest
emerges as a result of repeated interaction with content that is
affectively positive (as is the case with situational interest), that
contributes to the accumulation of stored knowledge and value, and that
requires personal investment of effort (Hidi \& Renninger, 2006). Thus
in exploring the role of youth engagement as a facilitator of interest
in summer STEM programs, we focus on these multiple dimensions of youth
engagement (more on this when we discuss engagement below).

Of course, no one enters summer STEM programs in an \enquote{interest
vacuum.} Indeed, many youth register for summer STEM programs precisely
because they already have an interest in STEM that they wish to develop
further: As such, summer STEM programs may support interest development
through a recursive processes. Alternatively, youth may become involved
in summer STEM programs for a variety of other reasons (i.e., because a
parent or caretaker decided they should do it; Beymer, Rosenberg, \&
Schmidt, 2018), in which case individual interest may not be a driving
force behind one's participation in a summer program. The point is,
youth enter summer programs with a certain amount of individual interest
in the content (whether high or low), which itself is the result of
prior opportunities for interaction with said content. To understand how
youth's level of engagement in summer STEM programs supports the
development of their individual STEM interest, it is necessary to
account for their individual level of interest at program entry.
Regardless of their level of STEM interest at program entry, youth who
are more deeply engaged during their time in the program may recursively
impact the level of their individual STEM interest (Hidi \& Renninger,
2006). As specified in theoretical accounts of the development of
individual interest, situational interest plays an important role in
their momentary interactions. Therefore, in order to focus on changes in
individual interest over the period of the program we include youths'
initial individual interest. To begin to account for the recursive
nature of interest development we also attempt to account for youth's
experienced situational interest during their daily engagement in
program activities.

The interest of any individual youth in STEM may also be influenced by
societal norms and stereotypes about what types of people are interested
in and qualified for particular domains. Despite the parity that exists
in males' and females' coursetaking and achievement in STEM areas, there
continue to be persistent stereotypes about many STEM fields as being
\enquote{male} fields, and of girls and women being less capable than
boys and men in STEM domains (see Hill, Corbett and St.~Rose, 2010 for a
review). These persistent stereotypes and the implicit and explicit
biases that emerge from them, can influence the way boys and girls
interact with STEM content, and their trajectories of interest
development over time. Given existing gender-related stereotypes about
STEM disciplines, it is important to take gender into account in our
examination. We also note that a similar argument might be made with
respect to youth race and ethnicity. Our analyses do not control for
this particular youth characteristic because nearly all of our study
participants are black and/or Hispanic and would be considered
underrepresented minorities in STEM fields.

\subsection{Measuring and Understanding Youth
Engagement}\label{measuring-and-understanding-youth-engagement}

Contemporary frameworks for engagement highlight the multiple dimensions
of of engagement and its dynamic nature due its malleability that can
change based on individual characteristics and pre-dispositions, as well
as environmental factors (Shernoff \& Schmidt, 2008; Skinner \& Pitzer,
2013). Most commonly, scholars refer to cognitive, affective, and
behavioral components (Christenson et al., 2012; Fredricks et al., 2004)
-- dimensions that correspond to the processes proposed to be involved
in interest development (see Hidi \& Renninger, 2006). Behavioral
engagement refers to one's involvement in activities in terms of their
effort and participation. Cognitive engagement refers to one's mental
investment in his or her own learning. Affective engagement refers to
one's positive and negative feelings one has toward learning activities
(Fredricks et al., 2004, 2011; Sinatra et al., 2015).

Engagement has been linked to a number of critical outcomes including
persistence, achievement, and interest (Sinatra et al., 2015).
Importantly, engagement has been shown to vary over time and context
(Shernoff \& Schmidt, 2008). For example, certain activities may lead to
higher or lower levels of engagement, which in turn may impact the
development of interest. Considering the wide variety of activities that
characterize both formal and informal learning settings, it is important
to measure engagement repeatedly over time and consider engagement in
situ, using data collection methods (e.g., experience sampling) that
allow for examining repeated measures over time. Examining engagement as
situational affords the opportunity to ultimately identify factors
influencing engagement that are under the control of educators (Schmidt
et al., 2018).

Scholars have developed methods to assess engagement in a way that makes
it possible to understand its changes and dynamics over time and across
activities. Among the approaches for measuring engagement in such a way,
intensive longitudinal methods that use Experience Sampling Method and
diary studies have shown their utility (Bolger \& Laurencau, 2013;
Hektner, Schmidt, \& Csikszentmihali, 2007). The present study adds to
this body of literature investigating how sustained engagement over a
period of time in summer STEM programs may have cumulative effects on
individual outcomes (interest) as measured before and after the
programs.

\section{The Present Study}\label{the-present-study}

This study explores how interest develops as a result of youth momentary
engagement measured using the experience sampling method (Hektner et
al., 2007). The research question driving this study is, How is youth's
sustained engagement across three weeks of summer STEM programming
related to changes in their STEM interest over the duration of the
program? The relationships examined in this paper are represented in
Figure 1. To address this research question,we use multiple data
collection methods to examine the development of interest over time via
youth engagement across a number of summer STEM programs.

\begin{figure}
\centering
\includegraphics{conceptual-framework.png}
\caption{conceptual framework}
\end{figure}

Figure 1. Conceptual framework for how engagement during summer STEM
programs impacts youths' development of interest in STEM domains.

\section{Context}\label{context}

The present study was part of a larger study focused on the engagement
and interest of youth while participating in one of nine summer STEM
programs in the Northeast United States. These nine summer programs are
part of two larger organizations that provide opportunities to
low-income youth to participate in summer programs where cost is not a
barrier due to the support from external and internal funding. While
attending the programs, adolescents participate in activities designed
to create interest and engagement in STEM. These programs are designed
so that adolescents spend about half of their time participating in
activities in the classroom and half of their time participating in
field experiences. These field experiences take place with community
partners where youth engage in activities including designing computer
games, exploring the ecology of islands, and planting community gardens.
Each program lasted between four and six weeks. Throughout those weeks,
adolescents participated Monday through Thursday for three hours each
day.

\section{Participants}\label{participants}

Participants consist of 203 youth (50\% female). The average age of
participants was 12.71 (SD = 1.18) and ranged from 10 to 16 and the
demographic makeup was 6\% White, 36\% Black, 48\% Hispanic, 7\% Asian,
and 3\% multiracial (as presented in Table 1).

Table 1. Demographic characteristics of youth.

\subsection{Procedure}\label{procedure}

Prior to the beginning of the program, youth completed a self-report
survey for collecting demographic information and their interest in
STEM. Over the course of the each program, data were collected on their
overall engagement using the experience sampling method (Hektner et al.,
2007). Data were collected using the experience sampling method for
three weeks of the program during two days each week. Data collection
took place so that field experiences and classroom experiences were
equally represented. The average attendance rate across programs was
83.08\% (SD = 0.16). Youth completed a second self-report survey at the
end of the program to assess their post-program interest in STEM.

The experience sampling method is an intensive longitudinal method of
data collection that is used to assess real-time experiences in
responded to randomly emitted signals (see Hektner et al., 2007 for a
description of this method). Adolescents were signalled four times per
day throughout six days of the program. The only condition for
signalling was that each signal must occur at least 15 minutes apart.
Adolescents responded to each signal via mobile phones that they were
given at the beginning of the day. Overall, 2,968 experience sampling
responses were collected (M = 14.6) and the completion rate was 63\%.
Just over half of all missed signals were due to youth absence from the
program.

\subsection{Measures}\label{measures}

\subsubsection{Engagement}\label{engagement}

Indicators of youth engagement were collected using the experience
sampling method. A composite measure was formed consisting of four items
(alpha = 0.85). Consistent with theoretical models of how sustained
individual interest develops, our measure of engagement includes an
indicator of momentary situational interest, which captures the extent
to which particular program activities engaged youth. The four items
asked youth to indicate on a four-point Likert scale (1 = not at all, 4
= very much) their agreement with the following statements:

\begin{itemize}
\tightlist
\item
  Working hard: As you were signaled, how hard were you working?
\item
  Concentrating: As you were signaled, how well were you concentrating?
\item
  Enjoying: As you were signaled, did you enjoy what you are doing?
\item
  Interest: Was the main activity interesting?
\end{itemize}

\subsubsection{Individual Interest}\label{individual-interest}

At the outset and conclusion of each program, a 4-point Likert Scale (1
= not at all, 4 = very much) was used to assess youth's individual
interest in the STEM area or areas that were the focus of the program
they enrolled in. For each relevant area, youth responded to three
questions about their interest:

\begin{itemize}
\tightlist
\item
  I have always been fascinated by science/math/building
\item
  I am interested in Science/math/building
\item
  At school, science/math/building things is fun
\end{itemize}

The individual interest measure represented the mean of interest items
across all relevant domains. Thus for some students, the mean was based
on 3 items, while for others it was based on as many as 9 items
representing all three domains (with Cronbach alpha values ranging from
.77 - .86 for each domain specific interest scale).

\subsection{Data Analysis}\label{data-analysis}

The goal of the data analysis was to understand how youths'
in-the-moment engagement, relates to their post-program STEM interest,
accounting for their pre-program STEM interest and gender. In order to
carry out this analysis, we conduct a two-step modeling approach. The
first step uses mixed effects (or multi-level) models using the lme4
(Bates, Machler, Bolker, \& Walker, 2015) package in R (R Core Team,
2018) to estimate the youth-specific mean levels of engagement across
all of their ESM responses, accounting for the effects of their
pre-program interest and gender (Model 1A). Then, we use the predicted
values for engagement as variables in the second model to predict
youths' post-program STEM interest, controlling for their pre-program
interest and with gender as a predictor (Model 1B). These predicted
values are called Best Linear Unbiased Predictors (BLUPs), and represent
a compromise between a) simply calculating the mean for each student and
b) calculating the overall level of engagement for all students,
weighting how much systematic information is available for each youth to
estimate a youth-specific effect. They are helpful as a way to concisely
summarize students' engagement over time, as measured through repeated
measures ESM responses. This approach is superior to simply calculating
the mean, which likely introduces bias into the estimation of models
that use it (i.e., results that are statistically significant may be
spurious (Gelman \& Hill, 2007). For youth j and ESM response i, the
models are:

\[
<!-- add model -->
\]

Because this study uses an observational design, we use sensitivity
analysis to quantify how robust our inferences are in light of
potentially omitted, confounding variables, and other potential sources
of bias (see Frank, 2000, for a description of the approach). Using the
konfound R package (Rosenberg, Xu, \& Frank, 2018), we describe what
percentage of an effect would need to be due to invalidate inferences
for effects.

The mixed effects modeling approach does not account for the uncertainty
in the predictions of youths' ESM engagement when used to predict their
post-program interest (Houslay \& Alastair, 2017). Accordingly, we were
concerned about the possibility of the results appearing stronger than
they should be. Therefore, we pursued a one-step modeling approach. We
used a Markov Chain Monte Carlo (MCMC) approach in a Bayesian framework,
estimating a model predicting youths' ESM engagement and post-interest
as part of a single model, with the relationship between these two
outcomes (after accounting for youths' pre-program interest and gender)
being used to explain how their ESM engagement related to changes in
their post-program interest.

MCMCglmm methods, unlike ML, requires priors. One view of these priors
is that they constrain the possible values that parameters may take in
order to set the modeling up for success (Houslay \& Wilson, 2017).
Another way to look at priors is to consider them as part of a Bayesian
approach, in which they represent the degree of belief in different
parameter values (Gelman \& Hill, 2007). There are also cases when the
prior values can be estimated from the data in the sample. Gelman and
Hill (2007) describe multi-level models in these terms: For the
\enquote{random} effects, usually \enquote{grouping} variables like the
classroom students are in, for example, the prior for the
classroom-specific effects is estimated on the basis of the mean and
variance in the dependent variable from the whole sample or data set
collected. In these cases (in which the prior for \enquote{random}
effects can be estimated from the data), the priors for the other
variables can be set to be neutral, with the aim to constrain the
possible values that parameters may take rather than to be used as part
of a fully Bayesian approach. Accordingly, for the models specified for
this study, we specified weak priors that are consistent with a wide
range of possible parameter values. Some authors refer to these as
\enquote{uninformative} priors; for identical model specifications, the
results from use of uninformative priors with MCMCglmm and ML are
identical. When we run the MCMC procedure, we follow guidelines by
Kruschke (2015) for checking the representativeness (by checking the
burn-in period and the convergence of multiple chains), accuracy (by
checking the effective sample size and the the Monte Carlo standard
error (MCSE) for estimates), and efficiency of the estimation (by
running multiple chains in parallel and by way of using the efficient
MCMCglmm package.

\section{Results}\label{results}

\subsection{Preliminary results}\label{preliminary-results}

First, we report descriptive statistics and Pearson correlations for all
study variables. Table 2 shows that youths' pre and post-program
interest had similar means (pre-interest: M = 3.04 (SD = 0.90);
post-interest: 3.10 (0.91), measured on a scale with a range of
one-four, indicating higher-than-average interest in STEM. The mean
difference (Mpre-interest - Mpre-interest = 0.03) in interest, using a
paired t-test, was not significant (t =0.485, p = 628), indicating
that--not accounting for youth' engagement during the programs--youths'
average interest does not change from before to after their involvement
in the program. The correlation between pre and post-interest (r = .59)
was high, indicating that youth who enter the program with high interest
are likely to leave with high interest. Note that this does not suggest
that these youths' interest has changed, but rather that there is a high
degree of alignment between youths' pre- and post-interest in STEM
domains. Youths' mean level of engagement was 2.86 (0.86), indicating
moderately high youth engagement as measured by the approximately 2,700
ESM responses youth completed throughout the programs. The correlation
between ESM engagement and pre-interest and post-interest was moderate
(r = .15 and .30, respectively).

\subsection{Mixed effects models
results}\label{mixed-effects-models-results}

First, we present results from mixed effects (or multilevel) models.
Model 1A specified youths' momentary engagement, measured via the ESM,
as the outcome, with fixed effects for youths' gender and pre-program
STEM interest, and a random intercept--representing youth-specific
differences from the overall, mean levels of engagement--for each youth.
The first model, predicting youths' ESM engagement (their engagement as
indicated by their responses over the course of their time in the
programs) show that females' engagement is estimated to be lower than
males' (B = -.06 (SE = .09), p = .45). But, this effect did not reach
the criterion for statistical significance. The effect of youths'
pre-program interest was positive and statistically significant (B =
0.10 (0.05), p = .033), suggesting that youth with relatively higher
levels of pre-program interest tended to be more engaged during daily
program activities. However, only 6.81\% of this inference would need to
be due to bias to be invalidated, indicating that this effect is likely
not very robust due to potential sources of bias. Model 2B predicted
students' post-program interest on the basis of their gender,
pre-program interest, and their predicted engagement (which is a
youth-specific prediction of their engagement throughout their time in
the program). The results for this outcome showed that female interest
is estimated to be lower than for males (B = -0.14 (SE = 0.11), p =
.239), though it did not reach statistical significance. Not
surprisingly, the effect of youths' pre-program interest upon
post-program interest is positive and statistically significant (B =
0.62 (SE = 0.06), p \textless{} .001) indicating that youth with high
levels of pre-program interest tend to have high levels of post-program
interest when they leave the program This effect was robust, with
78.27\% needing to be due to bias in order to be invalidated. The effect
of youths' predicted engagement during the course of programming upon
their post-program interest was positive and statistically significant
(B = 0.46 (SE = 0.10), p \textless{} .001), even after accounting for
the effects of their pre-program interest and gender. This suggests that
youth who sustain high levels of momentary engagement over the course of
the program tend to experience increases in STEM interest over the
course of the summer program, regardless of their level of STEM interest
when the program began. This effect was also robust, with 53.34\%
needing to be due to bias for it to be invalidated. We also calculated a
partial R2 for this effect, finding that it indicates a moderate-sized
effect (partial R2 = .340).

\subsection{MCMC results}\label{mcmc-results}

add!

\section{Discussion}\label{discussion}

\subsection{Key Findings}\label{key-findings}

We sought to understand how youths' engagement in summer STEM programs
impacted their development of interest. Instead of using youths' mean
level of engagement across their ESM responses, we first used BLUPs from
mixed effects models by modeling ESM engagement as an outcome, which
were estimated along with the other parameters estimated in the first
model. For this model, we found that youths' pre-program interest
predicted their in-the-moment engagement (though the effect was small),
suggesting that youth who begin summer STEM programs with higher initial
interest have a (modest) tendency to be more engaged while attending the
program. Gender did not predict youths' in-the-moment engagement.
Second, we used the BLUPs (as students' predicted interest over their
time in the program) as predictors in the model with youths'
post-program interest as the outcome. In alignment with Hidi and
Renninger's (2006) model of interest development and in support of
research (i.e., Dabney et al., 2012) on the impact of OST STEM programs,
we found that students' sustained engagement over the course of the
summer programs strongly predicted changes in their interest: For every
one-unit change in their predicted engagement (when engagement is
measured on a one-four scale), interest in STEM (also measured with a
one-four scale) after the program was .46 units greater.

Particularly in light of the overall lack of change in youths' interest
(the not statistically significant change in pre-post program interest
in STEM), these results show that is not participation in summer STEM
programing, but level of engagement in these program impacts their
interest in STEM domains. These results suggest that youths' experiences
in summer STEM programs strongly and positively relate to changes in
interest in STEM domains.

Notably, when we used a more conservative MCMC approach (Appendix B),
the effects were essentially the same, though with a slightly smaller
estimated effect for the relationships observed. For example, the
partial R2 (calculated to more easily compare results between the two
approaches) for the effect of youths' predicted engagement for the
effect of youths' predicted engagement was moderate--and statistically
significantly different from zero--using both modeling approaches
(Partial R2 from the mixed effects modeling approach = .34; Partial R2
from the mixed effects modeling approach = .27). While both approaches
yielded similar results in this case, in cases with less strong effects,
the more conservative MCMC approach could be used to avoid making
potentially spurious inferences (Houslay \& Wilson, 2017). Particularly
for complex data structures, such an approach can compare and in some
cases have advantages (i.e., in data with complex random effects
structures) over other approaches, such as multilevel Structural
Equation Modeling. While this was a study designed to ask and answer a
substantive question about youths' engagement and how it relates to
their interest, it demonstrates the use of a data analytic approach
(i.e., the use of youths' predicted engagement from mixed effects
models--and, in the appendix, models estimated with MCMC) that is
particularly suited to exploring how in-the-moment experiences relate to
their antecedents and outcomes. While we used ESM data, this approach
could also be used for other methods of data collection, such as
log-trace data, such as that from intelligent tutoring systems and
online microworlds (Gerard, Ryoo, McElhaney, Liu, Rafferty, \& Linn,
2015; Gobert, Baker, \& Wixon, 2015).

\subsection{Limitations and recommendations for future
research}\label{limitations-and-recommendations-for-future-research}

While the use of ESM is a strength of the present study, one limitation
concerns the specific items used and how we used them to measure youths'
engagement. We measured engagement using a composite, aiming to tap the
multiple dimensions of engagement in past research, but not
distinguishing them in the analysis. As a consequence, we do not
understand whether the changes are a consequence of behavioral,
cognitive, and affective engagement (which includes an indicator of
situational interest)--or some particular configuration of these
dimensions. Future research can better explore which types of engagement
matter in which ways in terms of their impact on youths' interest. To
this end, person-oriented analyses, and analytic approaches such as
Latent Profile Analysis, hold promise for understanding how the
dimensions of engagement are experienced by youth. Researchers can also
explore other outcomes, such as changes in youths' future goals and
plans and their competence.

Another limitation concerns the sample of participating youth in the
present study. The sample was collected in a highly purposive manner,
such that programs designed and implemented with best practices for OST
STEM programming were identified and selected as the context for this
study. While the results use students' pre-interest to account for how
their initial inclination toward STEM domains impacts their post-program
interest, their experiences during the programs are highly contingent
upon the activities they are involved in and the quality of the guidance
and instruction of the youth activity leaders. Additionally, the sample
was made up of youth almost completely from underrepresented (in STEM)
groups of individuals. Accordingly, we are not yet able to say how these
findings might generalize to other summer STEM programs, and considering
how youths' engagement impacts their interest in other, different OST
contexts may be a worthwhile aim of future research. Related to this
limitation, there was a large degree of missing data, though not
substantially higher than that in other studies that use ESM (Hektner et
al., 2007). Our use of sensitivity analysis showed that more than 50\%
of the effect of youths' engagement upon the changes in their interest
would need to be due to bias for it to be invalidated, suggesting that
issues related to missing data would have to be very substantial (i.e.,
the effect of survivorship bias, in that only the youth who were most
interested attended and completed all of the survey measures) for this
effect to be negated were the data that are missing to be included.
Nevertheless, particularly in OST settings, where data collection can
present some challenges, future research can carefully consider and aim
to mitigate potential impacts of missing data.

\section{Conclusion}\label{conclusion}

The aim of this study was to explore whether and how youths' engagement
in summer STEM programs promotes their development of interest in STEM.
Using ESM as a data collection methodology and a unique mixed effects
modeling approach for the data analysis, we found that youths'
experiences in terms of their in-the-moment engagement led to changes in
their post-program interest. The method of data collection (ESM) is an
example of how intensive data methods (i.e., Bolger \& Laurencau, 2013)
can be used to ask questions that are difficult to answer simply using
self-report surveys, which do not lend (as much) insight into the
experience of individuals during the moment, activity, or programming we
are interested in. These findings are in line with previous research
suggesting that out-of-school time STEM programs can lead to increased
STEM interest (Dabney et al., 2012; Elam et al., 2012). In this study,
we sought to add to this literature by examining one possible mechanism,
engagement, for the development of youth interest. Engagement is widely
understand by educators and scholars as important to many adaptive
academic outcomes, including interest. As out-of-school time programs
are well-suited to supplement STEM learning among youth with engaging
activities, it is important for researchers to continue to focus on
these informal learning environments as they are particularly promising
vehicles for interest development.

These findings have some implications for practice as well as the
research literature. The experiences that youth have in summer STEM
programs have an impact on their STEM interest after their involvement
in these programs; accordingly, youth activity leaders should seek to
design activities that are highly engaging to youth. Research on
engaging activities in summer STEM programs suggests expansion of
initiatives to provide opportunities for youth, particularly those with
less opportunities to develop interest in STEM domains, to participate
in out-of-school time STEM programs.

\newpage

\section{References}\label{references}

Bates, D., Maechler, M., Bolker, B., \& Walker, S. (201). Fitting Linear
Mixed-Effects Models Using lme4. Journal of Statistical Software, 67(1),
1-48. \url{doi:10.18637/jss.v067.i01}. Bell, P., Lewenstein, B., Shouse,
A.W., \& Feder, M.A. (Eds.) (2009). Learning science in informal
environments: People, places and pursuits. Washington, DC: National
Academy Press. Beymer, P. N., Rosenberg, J. M., Schmidt, J. A., \&
Naftzger, N. J. (2018). Examining Relationships among Choice, Affect,
and Engagement in Summer STEM Programs. Journal of Youth and
Adolescence, 1-14. Bolger, N., \& Laurenceau, J. P. (2013). Intensive
longitudinal methods. New York, NY: Guilford. Brophy, J. (2008).
Developing students' appreciation for what is taught in school.
Educational Psychologist, 43(3), 132-141.
\url{doi:10.1080/00461520701756511} Brophy, J. (2008). Developing
students' appreciation for what is taught in school. Educational
psychologist, 43(3), 132-141. Christenson, A. L. Reschly, \& C. Wylie
(Eds.), The handbook of research on student engagement (pp.~763--782).
New York: Springer Science.
\url{https://doi.org/10.1007/978-1-4614-2018-7_37} Fayer, S., Lacey, A.,
\& Watson, A. (2017). STEM occupations: past, present, and future.
Frank, K. 2000. \enquote{Impact of a Confounding Variable on the
Inference of a Regression Coefficient.} Sociological Methods and
Research, 29(2), 147-194
\url{https://msu.edu/~kenfrank/papers/impact\%20of\%20a\%20confounding\%20variable.pdf}
Fredricks, J. A., Blumenfeld, P. C., \& Paris, A. H. (2004). School
engagement: Potential of the concept, state of the evidence. Review of
Educational Research, 74(1), 59-109. Gelman, A., \& Hill, J. (2007).
Data analysis using regression and multilevel, hierarchical models (Vol.
1). New York, NY, USA: Cambridge University Press. Gerard, L. F., Ryoo,
K., McElhaney, K. W., Liu, O. L., Rafferty, A. N., \& Linn, M. C.
(2016). Automated guidance for student inquiry. Journal of Educational
Psychology, 108(1), 60. Gobert, J. D., Baker, R. S., \& Wixon, M. B.
(2015). Operationalizing and detecting disengagement within online
science microworlds. Educational Psychologist, 50(1), 43-57. Hadfield,
J. D. (2010). MCMC methods for multi-response generalized linear mixed
models: The MCMCglmm R Package. Journal of Statistical Software, 33(2),
1-22. \url{http://www.jstatsoft.org/v33/i02/}. Harackiewicz, J. M.,
Smith, J. L., \& Priniski, S. J. (2016). Interest matters: The
importance of promoting interest in education. Policy Insights from the
Behavioral and Brain Sciences, 3(2), 220-227. Hektner, J. M., Schmidt,
J. A., \& Csikszentmihalyi, M. (2007). Experience sampling method:
Measuring the quality of everyday life. Sage. Hidi, S. \& Renninger, K.
A. (2006). The four-phase model of interest development. Educational
Psychologist, 41(2), 111-127. Hidi, S., Renninger, K.A., \& Krapp, A.
(2004). Interest, a motivational variable that combines affective and
cognitive functioning. In D. Y Dai \& R.J. Sternber (eds.), Motivation,
emotion and cognition: Integative perspectives on intellectual
functioning and development (pp.~89-115). Mahwah, NJ: Lawrence Erlbaum.
Hill, C., Corbett, C., \& St Rose, A. (2010). Why so few? Women in
science, technology, engineering, and mathematics. American Association
of University Women. 1111 Sixteenth Street NW, Washington, DC 20036.
Hirsch, B. J., Mekinda, M. A., \& Stawicki, J. (2010). More than
attendance: The importance of after-school program quality. American
Journal of Community Psychology, 45(3), 447--452.
\url{https://doi.org/10.1007/s10464-010-9310-4}. Houslay, T. M., \&
Wilson, A. J. (2017). Avoiding the misuse of BLUP in behavioural
ecology. Behavioral Ecology, 28(4), 948-952. Lauer, P. A., Akiba, M.,
Wilkerson, S. B., Apthorp, H. S., Snow, D., \& Martin-Glenn, M. L.
(2006). Out-of-school-time programs: A meta-analysis of effects for
at-risk students. Review of educational research, 76(2), 275-313.
National Academy of Engineering and National Research Council (2014).
STEM integration in K-12 Education. Washington, DC: National Academies
Press. OECD (2016), PISA 2015 Assessment and Analytical Framework:
Science, Reading, Mathematics and Financial Literacy, OECD Publishing,
Paris. \url{http://dx.doi.org/10.1787/9789264255425-en} Prenzel, M.
(1992). Selective persistence of interests. In K.A. Renninger, S. Hidi
\& A. Krapp (eds.), The role of interest in learning and development
(pp.~71-98). Hillsdale, NJ: Lawrence Erlbaum. R Core Team (2018). R: A
language and environment for statistical computing. R Foundation for
Statistical Computing, Vienna, Austria. URL
\url{https://www.R-project.org/}. Rosenberg, J. M., Xu, R., \& Frank, K.
A. (2018). Konfound-It: Interactive application (and Stata procedure and
R package) to carry out sensitivity analysis.
\url{http://konfound-it.com}. Shernoff, D. J., \& Schmidt, J. A. (2008).
Further evidence of an engagement--achievement paradox among US high
school students. Journal of Youth and Adolescence, 37(5), 564-580.
Sinatra, G. M., Heddy, B. C., \& Lombardi, D. (2015). The challenges of
defining and measuring student engagement in science. Educational
Psychologist, 50(1), 1-13. \url{doi:10.1080/00461520.2014.1002924}
Skinner, E., Pitzer, J., \& Steele, J. (2013). Coping as part of
motivational resilience in school: A multidimensional measure of
families, allocations, and profiles of academic coping. Educational and
psychological measurement, 73(5), 803-835.






\end{document}
